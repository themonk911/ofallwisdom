\documentclass[a4paper]{article}

\usepackage[english]{babel}
\usepackage[utf8]{inputenc}

\usepackage{amsmath}
\usepackage{hyperref}


%%%%%%%%%%%%%%%%%%%%%%%%%%%%%%%%%%%%%%%%%%%%%%
\begin{document}

\title{Computable AIXI}
\author{\textsf{MIRIxCanberra}}
\date{6--7 September 2014}

\maketitle

\begin{abstract}
This workshop is on \emph{computable versions of AIXI}.
In particular, we investigate a computable $\xi$,
obtained by replacing Kolmogorov complexity with $Kt$ complexity.
The ultimate goal is to bring self-reflection to the AIXI formalism.
\end{abstract}


%%%%%%%%%%%%%%%%%%%%%%%%%%%%%%%%%%%%%%%%%%%%%%
\section{Introduction}

AIXI considers only computable hypotheses
but is itself incomputable,
so cannot consider itself part of the environment.
We expect that AIXI can learn and reason about parts of itself,
and in this sense that it can do some kind of self-reflection.
However, the AIXI model is not natural for reasoning about
the self-reflection that AIXI might do.
Because of this, we desire a different version of AIXI that
allows self-reflection explicitly.
A computable version of AIXI might be progress in this direction,
since it does not exclude itself from the hypothesis space \emph{in principle}.

In this workshop,
we want to find reasonable weights $w_\nu$
in the universal semimeasure $\xi$ that make it computable.
We do this by replacing Kolmogorov complexity with
the complexity measure $Kt$:
\[
   Kt(x \mid y)
:= \min \{ \ell(p) + \log t \mid p(y) = x \text{ halting in the $t$-th time step} \}.
\]
Choosing the weights $w_\nu = 2^{-Kt(\nu)}$,
we get the semimeasure
\[
\xi^{Kt}(x \mid y) := \sum_{\nu \in \mathcal{M}} w_\nu \nu(x \mid y).
\]
Using $\xi^{Kt}$ instead of $\xi$ yields a model that we call \emph{AIXI-$Kt$}.


%%%%%%%%%%%%%%%%%%%%%%%%%%%%%%%%%%%%%%%%%%%%%%
\section{Dates \& Location}

\begin{tabular}{ll}
Dates: & Sat 6 Sep and Sun 7 Sep 2014, starting 10:00{\sc am}.\\
Location: & N101 in CSIT \\
Note: & The workshop runs all day and meals will be provided.
\end{tabular}


%%%%%%%%%%%%%%%%%%%%%%%%%%%%%%%%%%%%%%%%%%%%%%
\section{Research Questions}

The following is a tentative list of research questions.
\begin{itemize}
\item Is $\xi^{Kt}$ a universal semimeasure? % Marcus: no
\item Does AIXI-$Kt$ satisfy some of the optimality properties of AIXI:
	Pareto optimality, balanced Pareto optimality, self-optimizing?
	% Marcus: yes, yes, no
\item What environments does AIXI succeed on but AIXI-$Kt$ does not?
	% Marcus: unclear what success means
\item What is the time and space complexity of AIXI-$Kt$?
\item What if we replace $\log t$ with some other function of $t$?
\item Are other AIXI-$Kt$ agents in AIXI-$Kt$'s hypothesis space?
	% Marcus: no
\item Can we find other computable weights $w_\nu$
	that satisfy more of the desirable properties
	mentioned in the questions above?
\end{itemize}


%%%%%%%%%%%%%%%%%%%%%%%%%%%%%%%%%%%%%%%%%%%%%%
\section{Background Readings}

In descending order of relevance:
\begin{itemize}
\item Introduction of $Kt$ and Universal Search~\cite[Ch.\ 7.5]{LV:2008}.
\item Optimality of AIXI~\cite[Ch.\ 5.4]{Hutter:2005}.
\item Resource bounded complexity in general~\cite[Ch.\ 7.1]{LV:2008}.
\item A related approach for Solomonoff Induction is
	the \emph{Speed Prior}~\cite{Schmidhuber:2002}.
\item Another computable approximation to AIXI is AIXI$tl$~\cite[Ch.\ 7.2]{Hutter:2005}.
\end{itemize}

% Kt paper: http://www.mathnet.ru/php/archive.phtml?wshow=paper&jrnid=ppi&paperid=914&option_lang=eng


%%%%%%%%%%%%%%%%%%%%%%%%%%%%%%%%%%%%%%%%%%%%%%
\bibliographystyle{alpha}
\bibliography{../marcus/ai}


\end{document}
