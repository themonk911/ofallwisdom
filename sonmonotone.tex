\section{The speed prior on a monotone Turing machine}

The speed prior is defined on non-monotone universal turing machines. We will show that it is straightforward to define it on a monotone turing machine, and in fact that the functions coincide. 

A monotone turing machine has three situations given some input $x$ and output $y$,


\begin{definition}[{Monotone Turing Machine~\cite[Def.\ 4.5.2 \& Def.\ 4.5.3]{LV:2008}\cite[Def.\ 2.6]{Hutter:2005}}]
\label{def:monotone-TM}
A \emph{monotone Turing machine}
is a Turing machine with
one unidirectional read-only input tape,
one unidirectional write-only output tape,
a finite number of work tapes, and no final states.
A monotone Turing machine implements a function $q$
that maps $x \in \X^\sharp$ to $y \in \X^\sharp$:
the input tape is initialized with $x$
and $y$ is read from the output tape according to the following cases.
\begin{enumerate}[(i)]
\item $x \in \X^*$ is finite and
	$y \in \X^*$ is to the left of the output tape's head when
	the head of the input tape reads the next character right of $x$.
\item The head of the output tape writes $y \in \X^*$ but no more
	as the machine runs forever where
	$x$ is infinite or
	the head on the read-only input tape never reads any characters right of $x$.
\item The machine writes $y \in \X^\omega$ to the output tape
	as it runs forever where
	$x$ is infinite or
	the head on the read-only input tape never reads any characters right of $x$.
\end{enumerate}
\end{definition}

The speed prior is defined as follows, 

$$ S(x) := \sum_{i=1}^\infty 2^i S_i(x) where S_i(x) = \sum_{p_i \rightarrow x} 2^{-l(p)}$$

We note that $p\rightarrow x$ means our program prefix $p$ reads p and computes output starting with $x$, while no prefix of $p$ consisting of less than $l(p)$ bits outputs $x$.

This definition means that programs whose prefixes have already computed $x$ do not contribute to $S(x)$ in later stages of FAST. This means that there are no fewer programs to sum over when we transfer to the monotone setting, since we consider a prefix-free set of programs to sum over.

Additionally, monotone Turing machines can accept strings with infinite input (see cases 2 and 3 above). However, in $S(x)$ we read in a finite prefix of the program and output a finite string as a result. We do this in stages of the algorithm FAST. This output string may be equal to $x$ at some point in which case the program prefix will contribute to the sum, but otherwise the infinite program will never contribute to the sum.

On a monotone Turing machine we replace $p \rightarrow x$ with $M(p'x) = y$