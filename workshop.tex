\documentclass[a4paper]{article}

\usepackage{amssymb}
\usepackage{enumerate}
\usepackage{hyperref}

\usepackage{mythm}
\usepackage{aixi}
\def\X{\mathcal{X}}

%opening
\title{Workshop}
\author{MIRIx people}

\begin{document}

\maketitle

\begin{abstract}
This is a loose compilation of notes from the MIRIx workshop at ANU on
September 6--7, 2014.
\end{abstract}

Things we have shown:\\

We spent a lot of time talking about $S$, the Speed prior~\cite{Schmidhuber:2002}.
$S$ is a computable semi-measure.
AI$S$ is like AI$\xi$ but with $S$ instead of $\xi$.\\
%
\begin{enumerate}
\item $S$ is not universal, because it is computable (and there is no universal computable prior).
\item $S$ is not a measure, for exactly the same reason that the Solomonoff prior is not a measure
	(some programs just print a finite string and stop).
\item However, we made up our own algorithm which does the same thing. Basically, it's clear that $S$ is lower semi-computable. So we show that $S(x)- S(x0) - S(x1)$ is also lower semi-computable, which leads to $S(x0)$ being computable.
\item $\epsilon$-optimal AIS is computable.
\item $S$ effectively dominates (see \autoref{def:effective-domination}) all lower semicomputable semimeasures.
\item $S(x) \leq 1 / |x|$
\item $S(x) \leq c_{S'} S'(x)$~\cite[Eq.\ 7]{Schmidhuber:2002}.
\end{enumerate}

\begin{definition}[Effective Domination]
\label{def:effective-domination}
A semimeasure $\nu$ \emph{effectively dominates} a semimeasure $\rho$ iff
there is a function $f: \mathbb{N} \to \mathbb{R}_+$ such that
\[
\nu(x) \geq 2^{-f(|x|)} \rho(x).
\]
\end{definition}

Effective domination implies absolute continuity on cylinder sets
(is it equivalent?).

We have that $S$ effectively dominates all lower semicomputable semimeasures,
but we are not sure whether this is useful at all.
For example, if we use $S$ for prediction of a lower semicomputable semimeasure $\mu$,
Hutter's loss bounds do not apply.
Instead we get
\begin{equation}
\label{eq:KL}
KL(\mu \dmid S) \leq - \ln w_\mu = f(|x|).
\end{equation}
In our case, $f(|x|) \to \infty$ as $|x| \to \infty$.
The proof for \eqref{eq:KL} is identical to \cite[Thm.\ 3.19]{Hutter:2005}
except for the application of effective domination in the last step.

For deterministic $\mu$,
this gives the error bound~\cite[Thm.\ 3.36]{Hutter:2005}
\[
E_t^{\Theta_S} \leq 2f(t),
\]
where $E_t^{\Theta_S}$ is the number of errors made up to time $t$
when using $S$ for prediction of the sequence generated by $\mu$.
This bound is only useful if $f(t) \in o(t)$.


\begin{definition}[{Monotone Turing Machine~\cite[Def.\ 4.5.2 \& Def.\ 4.5.3]{LV:2008}\cite[Def.\ 2.6]{Hutter:2005}}]
\label{def:monotone-TM}
A \emph{monotone Turing machine}
is a Turing machine with
one unidirectional read-only input tape,
one unidirectional write-only output tape,
a finite number of work tapes, and no final states.
A monotone Turing machine implements a function $q$
that maps $x \in \X^\sharp$ to $y \in \X^\sharp$:
the input tape is initialized with $x$
and $y$ is read from the output tape according to the following cases.
\begin{enumerate}[(i)]
\item $x \in \X^*$ is finite and
	$y \in \X^*$ is to the left of the output tape's head when
	the head of the input tape reads the next character right of $x$.
\item The head of the output tape writes $y \in \X^*$ but no more
	as the machine runs forever where
	$x$ is infinite or
	the head on the read-only input tape never reads any characters right of $x$.
\item The machine writes $y \in \X^\omega$ to the output tape
	as it runs forever where
	$x$ is infinite or
	the head on the read-only input tape never reads any characters right of $x$.
\end{enumerate}
\end{definition}


\bibliographystyle{alpha}
\bibliography{aixi}


\end{document}
