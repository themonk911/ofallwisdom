\documentclass[11pt]{amsart}
\usepackage{geometry}                % See geometry.pdf to learn the layout options. There are lots.
\geometry{letterpaper}                   % ... or a4paper or a5paper or ... 
%\geometry{landscape}                % Activate for for rotated page geometry
%\usepackage[parfill]{parskip}    % Activate to begin paragraphs with an empty line rather than an indent
\usepackage{graphicx}
\usepackage{amssymb}
\usepackage{epstopdf}
%opening
\title{Workshop}
\author{MIRIx people}

\begin{document}

\maketitle

Things we have shown:\\

We spent a lot of time talking about $S$, the Speed prior. $S$ is a computable semi-measure. AI$S$ is like AI$\xi$ but with $S$ instead of $\xi$.\\

\begin{enumerate}
\item $S$ is not universal, because it is computable (and there is not universal computable prior)
\item $S$ is not a measure, for exactly the same reason that the Solomonoff prior is not a measure
\item The algorithm AS from that paper is wrong, because it's not actually $\epsilon$-optimal. Counterexample: you have a small program that outputs $x$, and then a bunch of large programs which also output $x$. The algorithm halts prematurely and has a bad estimate.
\item However, we made up our own algorithm which does the same thing. Basically, it's clear that $S$ is lower semi-computable. So we show that $S(x)- S(x0) - S(x1)$ is also lower semi-computable, which leads to $S(x0)$ being computable.
\item $\epsilon$-optimal AIS is computable.
\end{enumerate}

\end{document}
